\section{Problemas encontrados}
En esta sección detallamos los problemas encontrados y los distintos acercamientos a sus soluciones.

\subsection{Manejo de errores}
El manejo de errores en EOF resultó tremendamente difícil. Si bien conceptualmente es fácil describir "error de comment no cerrado ocurre cuando se puede reducir BEGIN\_COMMENT comment\_words\_list eof" esto no tiene un mapeo simple al parser generator. Para el reporte de errores PLY ofrece el pseudo-símbolo error pero por alguna razón un error al final de un archivo no se reduce a este símbolo (o al menos no encontramos cómo lograrlo) para poder imprimir el mensaje de error que deseábamos.

Para manejar casos de error específicos cómo movimientos inválidos o comentarios ubicados en lugares incorrectos escribimos producciones que generan error al reducir. Éstas resultan fáciles de construir y mantener, pero acomplejizan el autómata y empeoran el manejo genérico de errores.

Para imprimir mejores mensajes de error en casos genéricos directamente escribimos un manejador de errores que imprime las transiciones válidas del autómata en el momento en el cuál se produjo el error. Esto puede ser un tanto confuso para un usuario pero resulta muy útil a la hora de debuggear. Al tener producciones que reducen a un error resulta falso que todas las transiciones "legales" "reducen bien". Si bien decidimos mantener ambos acercamientos al reporte de errores en nuestro proyecto creemos que esta incompatibilidad no es irresoluble, dado que las producciones de error podrían ser marcadas como tales y ser modificado el autómata para saber si un estado posee sólamente producciones de error en su conjunto de ítems.

\subsection{Visualización del autómata}
PLY ofrece un archivo parser.out con un detalle sobre el parser generado. Para poder diagnosticar varios de nuestros problemas y entender mejor el código generado creamos un par de scripts que transforman éste en un diagrama utilizando AWK y Graphviz.