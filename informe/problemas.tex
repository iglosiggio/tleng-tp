\section{Problemas encontrados}
En esta sección detallamos los problemas encontrados y los distintos acercamientos a sus soluciones.

\subsection{Manejo de errores}
El manejo de errores en EOF resultó ser uno de los problemas más dificiles encontrados.
Si bien conceptualmente es fácil describir algo con la forma ``error de comment no cerrado ocurre cuando se puede
reducir \textit{BEGIN\_COMMENT comment\_words\_list eof}'', esto no tiene un mapeo
simple al parser generator. Para el reporte de errores, PLY ofrece el
pseudo-símbolo \textit{error} pero por alguna razón un error al final de un archivo no
se reduce a este símbolo (o al menos no encontramos cómo lograrlo) para poder
imprimir el mensaje de error que deseábamos.

Para manejar casos de error específicos como movimientos inválidos o
comentarios ubicados en lugares incorrectos escribimos producciones que generan
error al reducir. Éstas resultan fáciles de construir y mantener, pero
acomplejizan el autómata y dificultan el manejo genérico de errores.

Para imprimir mejores mensajes de error en casos genéricos, implementamos
un manejador de errores que imprime las transiciones válidas del
autómata en el momento en el cuál se produjo el error. Esto puede ser un tanto
confuso para un usuario, pero resulta muy útil a la hora de debuggear. Al tener
producciones que reducen a un error, resulta falso que todas las transiciones ``legales'' reducen ``bien''

Finalmente decidimos mantener ambos acercamientos al reporte de errores en
nuestro proyecto. Sin embargo, creemos que esta incompatibilidad no es
irresoluble, dado que las producciones de error podrían ser marcadas como tales
y modificando el autómata para saber si un estado posee solamente
producciones de error en su conjunto de ítems.

\subsection{Visualización del autómata}
PLY ofrece un archivo parser.out con un detalle sobre el parser generado. Para poder diagnosticar varios de nuestros problemas y entender mejor el código generado creamos un par de scripts que transforman éste en un diagrama utilizando AWK y Graphviz. Éste es el diagrama que se puede ver al final de la sección de implementación.

\subsection{Creación de ejemplos}
Para probar el programa escribimos pequeños archivos de ejemplo que nos permiten evaluar los mensajes de error y los resultados del parsing. Una vez hecho esto los agregamos a un conjunto de tests que nos permitieron mejorar el trabajo evaluando en cada momento si nuestros cambios modificaban el compotamriento del programa de formas indeseadas. Quedó cómo trabajo futuro el utilizar nuestro programa con archivos PGN grandes disponibles en internet.
